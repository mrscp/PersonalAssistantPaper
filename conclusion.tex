In this paper, we presented the development of a Virtual Personal Assistant (VPA) developed with state of the earth technologies to enhance user interactions with digital world e.g.\ IoT devices.
VPAs have become a necessary companions by streamlining tasks and providing personalized assistance through voice-activated interactions.
Our proposed VPA system integrates speech-to-text conversion, chatbot capabilities, and text-to-speech synthesis, collectively designed to offer a versatile and user-friendly virtual assistant experience.

We have proposed to implement a Reinforcement Learning model for speech-to-text conversion.
So that the VPA can personalize its understanding of user-specific voice characteristics and serving individuals with diverse speech patterns, it can also adapt to those who are having disabilities.
After getting the speech into a text form then it passes that text into the chatbot, the chatbot functionality enables the assistant to engage in natural language processing, parse user intents, and provide relevant responses from the given context.
Furthermore, the text-to-speech synthesis enhances the user experience by generating human-like speech output.

The deployed VPA effectively handles Bengali speech input, utilizing the BanglaBERT model to accurately extract relevant information from provided contexts.
Currently, the system is not implemented to specific user needs to do the tasks, but its successful implementation validates the potential for future customization and making it a production grade system.

Provided a screenshot\ref{fig:result} to validate the system's functionality, illustrating a real-time dialogue interaction where the VPA comprehends and responds to Bengali speech input.
Overall, our work represents the combination of multiple technologies in creating a versatile VPA, which holds promise for a wide area of applications and user scenarios, especially for users with varying speech capabilities and accent preferences.
